\chapterinitial{The Bempp boundary element library} (originally named BEM++) started in 2011 as a project to develop an open-source C++ library for the fast solution of boundary integral equations. The original release came with a simple Python wrapper to the C++ library. Over time, more and more functionality was moved into the Python interface, while computationally intensive routines and the main data structures remained in C++.

This proved a burden when efforts started to modernize the library to be able to make better use of single-instruction-multiple-data (SIMD) optimization on CPUs and offload computation to GPUs. In 2018, we decided to rewrite the computational core from scratch. The aims were to support explicit SIMD optimization on CPUs with various instruction lengths, be able to offload computations to AMD, Intel, and Nvidia GPUs, and to base the complete codebase on Python. These aims naturally led to the choice of building a Python library based around OpenCL (using the PyOpenCL interface) and Numba.

At the end of 2019, we released the first version (0.1) of Bempp-cl \cite{Bempp-cl}. This was followed later in 2020 by version 0.2, the first release that we considered feature complete and mature for application use. Since then we have used Bempp-cl in a number of practical applications and many of our users are migrating to it from the old C++ based Bempp. In this article, we want to discuss the design choices behind Bempp-cl and provide a number of performance benchmarks on different compute devices, including CPUs, AMD, Intel, and Nvidia GPUs.

[To be continued with some more details about content and overview of the Sections]
