\chapterinitial{The Bempp boundary element library} (originally named BEM++) started in 2011 as a project to develop an open-source C++ library for the fast solution of boundary integral equations. The original release came with a simple Python wrapper to the C++ interface. Over time more and more functionality was moved into the Python interface. However, computationally intensive routines and the main data structures remained in C++. This proved a burden when efforts started to modernise the library to be able to make better use of Single-Instruction-Multiple-Data (SIMD) optimisation on CPUs and offload computation to GPUs. In 2018 we decided to rewrite the computational core from scratch. The aims were to support explicit SIMD optimisation on CPUs with various instruction lengths, be able to offload computations to AMD, Intel, and Nvidia GPUs, and to base the complete codebase on Python. These aims naturally lead to the choice of building a Python library based around OpenCL (using the PyOpenCL interface) and Numba. The name of the new library was Bempp-cl. At the end of 2019 we released an initial version 0.1 that we used to test the library in a number of real-world scenarios. This was followed later in 2020 by version 0.2, the first release that we considered feature complete and mature for application use. Since then we have used Bempp-cl in a number of practical application and many of our users are slowly migrating to it from the old C++ based Bempp. In this article we want to discuss the design choices behind Bempp-cl and provide a number of performance benchmarks on different compute devices, including CPUs, AMD, Intel, and Nvidia GPUs.

[To be continued with some more details about content and overview of the Sections]
